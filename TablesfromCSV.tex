	\documentclass{article}
	
	\usepackage{booktabs} % For \toprule, \midrule and \bottomrule
	\usepackage{siunitx} % Formats the units and values
	\usepackage{pgfplotstable} % Generates table from .csv
	
	% Setup siunitx:
	\sisetup{
	  round-mode          = places, % Rounds numbers
	  round-precision     = 2, % to 2 places
	}
	
	\begin{document}
	
	\begin{table}[h!]
	    \caption{Autogenerated table from .csv file.}
	    \label{table1}
	    \pgfplotstabletypeset[
	      multicolumn names, % allows to have multicolumn names
	      col sep=comma, % the seperator in our .csv file
	      display columns/0/.style={
	        column name=$Value 1$, % name of first column
	        column type={S},string type},  % use siunitx for formatting
	      display columns/1/.style={
	        column name=$Value 2$,
	        column type={S},string type},
		   display columns/2/.style={
	        column name=$Value 3$,
	        column type={S},string type},
		   display columns/3/.style={
	        column name=$Value 4$,
	        column type={S},string type},
		   display columns/4/.style={
	        column name=$Value 5$,
	        column type={S},string type},
		   display columns/5/.style={
	        column name=$Value 6$,
	        column type={S},string type},
	      every head row/.style={
	        before row={\toprule}, % have a rule at top
	        after row={
	            \si{\ampere} & \si{\volt}\\ % the units seperated by &
	            \midrule} % rule under units
	            },
	        every last row/.style={after row=\bottomrule}, % rule at bottom
	    ]{table.csv} % filename/path to file
	\end{table}
	
	\end{document}